\documentclass[a4paper, 12pt]{article}

\usepackage[english]{babel}
%\usepackage[portuges]{babel}
\usepackage[utf8]{inputenc}
\usepackage{amsmath}
\usepackage{indentfirst}
\usepackage{graphicx}
\usepackage{multicol,lipsum}
%\renewcommand{\figurename}{Figura}
\usepackage{hyperref}
\usepackage{rotating}
\usepackage{lscape}

\setlength{\parskip}{1em}

\begin{document}
%\maketitle

\begin{titlepage}
	\begin{center}
	
	%\begin{figure}[!ht]
	%\centering
	%\includegraphics[width=2cm]{c:/ufba.jpg}
	%\end{figure}

		\textbf{\Large{Easy recipes for an unskilled person}}\\
		\large{---}\\ 
		%\large{Programa}\\ 
		\vspace{15pt}
        \vspace{95pt}
        %\textbf{\large{Rascunho PEP}}\\
		%\title{{\large{Título}}}
		\vspace{3,5cm}
	\end{center}
	
	\begin{flushleft}
		\begin{tabbing}
			Kallil\\
	\end{tabbing}
 \end{flushleft}
	\vspace{1cm}
	
	\begin{center}
		\vspace{\fill}
			\today
			\end{center}
\end{titlepage}
%%%%%%%%%%%%%%%%%%%%%%%%%%%%%%%%%%%%%%%%%%%%%%%%%%%%%%%%%%%

% % % % % % % % %FOLHA DE ROSTO % % % % % % % % % %

% % % % % % % % % % % % % % % % % % % % % % % % % %
\newpage
\tableofcontents
\thispagestyle{empty}

\newpage
\pagenumbering{arabic}
% % % % % % % % % % % % % % % % % % % % % % % % % % %
\section{Salty Stuff}


\newpage
\section{Sweet stuff}

\subsection{Chocolate muffin}

Ingredients:

\begin{itemize}
	\item 2 Cups flour
	\item 1 Cup granulated sugar
	\item 1/2 cup unsweetened natural cocoa powder
	\item 1 teaspoon baking soda
	\item 1/2 teaspoon salt
	\item 1 and 3/4 cups semi-sweet chocolate chips
	\item 2 large eggs at room temperature
	\item 3/4 cup full fat sour cream or plain yogurt at room temperature
	\item 1/2 vegetable oil
	\item 1/2 whole milk at room temperature
	\item 1 and 1/2 teaspoons pure vanilla extract
\end{itemize}

How to:

\begin{enumerate}
	\item Preheat oven to 425°F (218°C). Spray a 12-count muffin pan with nonstick spray or use cupcake liners. This recipe yields about 14 muffins, so prepare a second muffin pan in the same manner or bake in batches and reserve leftover batter at room temperature for when the first batch is done.
	
	\item Whisk the flour, sugar, cocoa powder, baking soda, salt, and chocolate chips together in a large bowl. Set aside.
	
	\item Whisk the eggs, sour cream, oil, milk, and vanilla extract together until combined. Pour wet ingredients into dry ingredients and fold together with a rubber spatula or wooden spoon until completely combined. (Batter is quite thick, so I recommend a spatula or spoon over a whisk.) Avoid overmixing. The batter will be thick and sticky.
	
	\item Spoon the batter into liners, filling them all the way to the top. Bake for 5 minutes at 425°F then, keeping the muffins in the oven, reduce the oven temperature to 350°F (177°C). Bake for an additional 15-16 minutes or until a toothpick inserted in the center comes out clean. The total time these muffins take in the oven is about 20-21 minutes, give or take. (For mini muffins, bake 13-14 total minutes at 350°F (177°C) the whole time.)
	
	\item Cool muffins for 10 minutes in the pan, then transfer to a wire rack until ready to eat.
	
	\item Cover leftover muffins and store at room temperature for 5 days or in the refrigerator for 1 week.
\end{enumerate}

\newpage
\subsection{Lemon muffin}

Ingredients topping:

\begin{itemize}
	\item 1/4 cup flour
	\item 2 tablespoons
	\item 1 tablespoon unsalted butter at room temperature
\end{itemize}

How to topping:

\begin{enumerate}
	\item Make the streusel first: combine the ingredients and rub together with your fingers until the butter is incorporated and the mixture has a dry crunmbly texture. Set aside.
\end{enumerate}

\begin{center}
	\line(1,0){450}
\end{center}

Ingredients:

\begin{itemize}
	\item 2 large eggs
	\item 1/2 cup vegetable oil
	\item 3/4 cup granulated sugar
	\item zest of 2 lemons
	\item 1 teaspoon lemon extract (optional)
	\item 1/2 cup buttermilk
	\item 1/4 cup lemon juice
	\item 2 tsp baking powder
	\item 1/2 teaspoon baking soda
	\item 1/2 teaspoon salt
	\item 2 cups all purpose flour, using the fluff, scoop, and level method
\end{itemize}

How to:

\begin{enumerate}
	\item Preheat the oven to 375°F Line a muffin pan with paper liners.
	\item In a large mixing bowl whisk together the eggs, oil, and sugar, and zest until well combined.
	\item Whisk in the extract, buttermilk and lemon juice.
	\item Whisk in the baking powder, baking soda, and salt, then fold in the flour. Mix just until combined and no dry flour remains, the batter will be somewhat lumpy.
	\item Fill 11-12 muffins cups nice and full with the batter, and then top with the streusel.
	\item Bake for 20-23 minutes (mine took 23 minutes) until risen and a toothpick inserted in the center comes out without wet batter on it. Muffins are small and cook fast, so keep alert.
	\item Let the muffins cool for 5 minutes in the pan, then remove to a cooling rack.
	\item Be sure to store the cooled muffins in an airtight container to keep them fresh. They're best eaten the same day.
\end{enumerate}

\newpage
\subsection{Banana bread with chocolate chips}

Ingredients:

\begin{itemize}
	\item 3 ripe bananas, mashed with a fork
	\item 1/2 cup butter (1 stick) at room temperature
	\item 3/4 cup white sugar (or use 1 scant cup of honey)
	\item 2 eggs, room temperature
	\item 1 and 1/2 cups all-purpose flour
	\item 1 teaspoon baking soda
	\item 1/2 teaspoon salt
	\item 1/2 tsp real vanilla extract
	\item 1 cup chocolate chips	
\end{itemize}

How to:

\begin{enumerate}
	\item In the mixing bowl, cream together butter and sugar.
	\item Mix in mashed bananas and eggs.
	\item Whisk together dry ingredients: flour, salt, and baking soda and add to batter.
	\item Stir in vanilla, chocolate chips, transfer to prepared loaf pan and bake.
	\item 350°F in a preheat oven for 55 to 60 minutes or until a knife inserted in the center of the cake comes out clean.
\end{enumerate}

\newpage
\end{document}



